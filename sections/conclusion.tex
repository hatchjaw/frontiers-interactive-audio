\section{Conclusion}\label{sec:conclusion}

In this article we have described an exploration of transmission of audio data
over computer networks, distributed computing and audio spatialisation.
A networked audio system was developed, strategies devised for
addressing challenges presented by distributed computing, and a distributed
spatial audio system was developed and deployed.
Evaluation of the system exposed the extent of the technical challenges
that confront it in its current form, and shed light on opportunities for
further development.
Perceptual testing revealed that it may, with certain limitations, offer
performance sufficient to support timing-critical sound field synthesis
techniques.
The proposed system represents a milestone on the road towards an accessible
alternative to state of the art spatial and immersive audio installations.

It has been shown that a system of discrete computational entities can
communicate effectively over an ethernet network, exchanging audio and control
data in a scalable fashion via a UDP multicast group.
Though facing significant technical challenges, an approach to the foundational
requirements of such a system has been demonstrated, with encouraging results
with regard to spatial audio applications.

Timing discrepancies between entities in the networked audio system give rise
to audible artefacts such as time-varying comb-filtering.
Loss of synchronicity is difficult to measure and compensate for in real-time
and the extent of the resulting audible disturbances is unpredictable, but,
depending on the nature of the sound sources being dealt with, certainly
perceptible.
The developed strategies for mitigation of asynchronicity are
\textit{best-effort} in nature, and call for further refinement, or replacement
with more sophisticated techniques.

Given its potential to disrupt the present situation with regard to spatial
audio installations, or complement it with a modular approach that could serve
more flexible and perhaps creative ends, it is hoped that scope will exist to
develop this work further.
A number of technical challenges remain, and questions with regard to the
fundamental, low-level characteristics of the various components of the devised
system stand unanswered.
From the level of audio hardware and driver software, to audio host and audio
buffer behaviour, to network QoS and the performance of network switches, to
the behaviour of the Teensy platform, and the processor and audio codec that it
is built around, plus its software libraries \textemdash{} much that is
typically taken for granted in the development of embedded systems, and audio
and networking systems, calls for deeper investigation.

Each microcontroller generates a clock signal to deliver to its audio shield.
In a locally distributed setting, this is quite redundant.
It may be possible to generate a single, authoritative clock on one node and
deliver that to all others in the system.
If this can be achieved, it would remedy the issue of clock drift, leaving only
jitter to be addressed.
Furthermore, Teensy is not the only platform worthy of consideration;
developments in microcontroller technology are to be expected, and devices
supporting higher-quality audio than offered by Teensy may emerge.
One device not considered here is the Raspberry Pi, a powerful, low-cost
platform, most commonly operated as a single-board computer, with an operating
system.
Support exists for \textit{bare-metal} development for the Raspberry Pi,
however, and since it can output 24-bit audio, it should be investigated for its
suitability.

With regard to audio spatialisation, a basic, linear, wave field synthesis
algorithm has been demonstrated here, implementing primary virtual sources.
WFS is capable of producing other types of sources, supporting nonlinear
speaker arrays, three-dimensional arrays, and more faithful models of
energy loss due to absorption.
There are many further avenues to pursue, including optimising any DSP
algorithms to best exploit the capabilities of what, in the shape of the
Teensy and other microcontroller systems, are very powerful platforms.
Further, and only given cursory treatment here, higher order ambisonics, subject
to an assessment of its suitability to parallelisation, remains as a worthy
target for implementation in future work.
