\begin{abstract}

%%% Leave the Abstract empty if your article does not require one, please see
%%% the Summary Table for full details.
    \section{}
    For full guidelines regarding your manuscript please refer to
    \href{http://www.frontiersin.org/about/AuthorGuidelines}{Author Guidelines}.

    As a primary goal, the abstract should render the general significance and
    conceptual advance of the work clearly accessible to a broad readership.
    References should not be cited in the abstract.
    Leave the Abstract empty if your article does not require one, please see
    \href{http://www.frontiersin.org/about/AuthorGuidelines#SummaryTable}{Summary Table}
    for details according to article type.


    Systems for spatial audio typically demand large numbers of loudspeakers
    and audio hardware capable of serving many output channels.
    Centralised systems of this sort are inflexible, and, due to their reliance
    on specialist audio hardware and software, costly, with a high barrier to
    entry.
    Recent decades have seen increasing interest in both audio spatialisation
    and the transmission of audio over computer networks.
    Advancements in low-cost microcontroller platforms with support for
    networking and audio processing, may facilitate a decentralised approach to
    audio spatialisation systems.
    Based on one such platform, we describe the development of a modular,
    scalable system of distributed audio processors with applications to
    spatial audio.
% Distributing audio processing across a network of such devices potentially
% represents a modular, scalable alternative.
% In this thesis, the development of such a system is described.
    Though faced by significant technical challenges,
% particularly with regard to
% maintaining synchronicity between distributed audio processors,
    the system demonstrates interesting initial perceptual results.
    Findings are commensurate with a capability, with further development
    and research, to disrupt and democratise the fields of spatial and immersive
    audio.

    \tiny
    \keyFont{ \section{Keywords:}\label{sec:keywords:}
        keyword,
        keyword,
        keyword,
        keyword,
        keyword,
        keyword,
        keyword,
        keyword
    } %All article types: you may provide up to 8 keywords; at least 5 are mandatory.
\end{abstract}
