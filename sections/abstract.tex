\begin{abstract}

%%% Leave the Abstract empty if your article does not require one, please see
%%% the Summary Table for full details.
    \section{}

%    For full guidelines regarding your manuscript please refer to
%    \href{http://www.frontiersin.org/about/AuthorGuidelines}{Author Guidelines}.
%
%    As a primary goal, the abstract should render the general significance and
%    conceptual advance of the work clearly accessible to a broad readership.
%    References should not be cited in the abstract.
%    Leave the Abstract empty if your article does not require one, please see
%    \href{http://www.frontiersin.org/about/AuthorGuidelines#SummaryTable}{Summary Table}
%    for details according to article type.


    Audio spatialisation is a topic of considerable research interest, with
    applications to immersive and extended reality experiences.
    State-of-the-art spatial audio systems are typically very costly, however,
    being reliant on specialist audio hardware capable of performing
    computationally intensive signal processing and delivering output to many
    tens, if not hundreds, of loudspeakers.
    Centralised systems of this sort suffer from limited accessibility due to
    their inflexibility and expense.
    Building on the research of the past few decades in the transmission of
    audio data over computer networks, and the emergence in recent years of
    increasingly capable, low-cost microcontroller-based development platforms
    with support for both networking and audio functionality, we present a
    decentralised, modular alternative.
    Having previously explored the feasibility of running a microcontroller
    device as a networked audio client, here we describe the development of a
    client-server system with an emphasis on scalability via multicast data
    transmission.
    The system operates on ubiquitous, commonplace computing and networking
    equipment, with a view to it being a simple, versatile, and
    highly-accessible platform, capable of granting users the freedom to explore
    audio spatialisation approaches at much reduced cost and commitment.
    Though faced by significant technical challenges, particularly with regard
    to maintaining synchronicity between distributed audio processors,
    the system produces perceptually plausible results, and stands as an
    important step on the path to producing a system to complement or rival the
    state of the art.
    Findings are commensurate with a capability, with further development
    and research, to disrupt and democratise the fields of spatial and immersive
    audio.

%%%%%%%%%%%%%%%%%%%
%    Systems for spatial audio typically demand large numbers of loudspeakers
%    and audio hardware capable of serving many output channels.
%    Centralised systems of this sort are inflexible, and, due to their reliance
%    on specialist audio hardware and software, costly, with a high barrier to
%    entry.
%    Recent decades have seen increasing interest in both audio spatialisation
%    and the transmission of audio over computer networks.
%    Advancements in low-cost microcontroller platforms with support for
%    networking and audio processing, may facilitate a decentralised approach to
%    audio spatialisation systems.
%    Based on one such platform, we describe the development of a modular,
%    scalable system of distributed audio processors with applications to
%    spatial audio.
%% Distributing audio processing across a network of such devices potentially
%% represents a modular, scalable alternative.
%% In this article, the development of such a system is described.
%    Though faced by significant technical challenges, particularly with regard
%    to maintaining synchronicity between distributed audio processors,
%    the system produces perceptually plausible results.
%    Findings are commensurate with a capability, with further development
%    and research, to disrupt and democratise the fields of spatial and immersive
%    audio.

    \tiny
    \keyFont{ \section{Keywords:}\label{sec:keywords}
    Spatial audio,
        Networked audio,
        Distributed systems,
        Wave Field Synthesis,
        Microcontroller,
%        keyword,
%        keyword,
%        keyword
    } %All article types: you may provide up to 8 keywords; at least 5 are mandatory.
\end{abstract}
