\subsection{Hardware Platforms}\label{subsec:hardware-platforms}

\begin{table}[t]
    \centering
    \begin{tabular}{ c c c r }
        Platform &
        Processor &
        Memory &
        Price \\

        \midrule

        Teensy 4.1\tablefootnote{\url{https://pjrc.com/store/teensy41.html}} &
        ARM Cortex-M7 \qty{600}{\MHz} &
        \qty{1}{\mega\byte} SDRAM &
        \texteuro{31.50} \\

        Daisy Seed\tablefootnote{\url{https://electro-smith.com/daisy/daisy}} &
        ARM Cortex-M7 \qty{480}{\MHz} &
        \qty{64}{\mega\byte} SDRAM &
        \texteuro{28} \\

        ESP32-LyraTD-MSC\tablefootnote{\url{https://espressif.com/en/products/devkits/esp-audio-devkits}} &
        Xtensa LS6 \qty{240}{\MHz} &
        \qty{8}{\mega\byte} PSRAM &
        \texteuro{49.50} \\

        Bela\tablefootnote{\url{https://shop.bela.io/products/bela-starter-kit}} &
        ARM Cortex-A8 \qty{1}{\GHz}\tablefootnote{\url{https://beagleboard.org/black}} &
        \qty{512}{\mega\byte} DDR3 &
        \texteuro{209}
    \end{tabular}
    \caption{Comparison of selected embedded audio development platforms.
    Prices as of January 2024.}
    \label{tab:embedded-comparison}
\end{table}

The notion of taking a distributed approach to DSP is reliant on the
identification of a suitable supporting hardware platform.
For a distributed audio application, the ideal computing platform should be
small, inexpensive, plus easily and rapidly programmable;
of course, it should also provide audio and networking hardware, and, ideally,
well-documented APIs for interacting with this hardware.

Recent years have seen the emergence of a number of small, low-cost, open-source
platforms for embedded development, perhaps best known amongst these being
Arduino\footnote{\url{https://arduino.cc/}}.
Though support for audio is limited via official Arduino models, a number of
audio-specific, Arduino-like systems have been produced, such as various
ESP32 and STM32 models, Daisy Seed (itself an STM32-based board), and the Teensy
family of microcontroller development boards.
Also worthy of consideration is the Bela platform; though not a true
microcontroller (being based on the Beaglebone single-board computer, and
running an operating system) it is a small-footprint audio-focused device
suitable for embedded applications.
Platforms of this kind that also possess networking support may also fall
under the category of \textit{Internet of Things} (IoT) devices.

A comparison of selected devices can be found in
\tabref{tab:embedded-comparison}.
Bela is significantly more powerful than the microcontroller systems, but it is
commensurately costly.
Daisy Seed is well-appointed with memory (which is important for DSP algorithms
featuring long delay-lines, for example), but does not feature ethernet support.
Teensy4.1, and the selected ESP32 device support networking via their
respective ethernet add-ons, but the latter's CPU is under-powered.
Though lacking in memory, Teensy's processor and low price make it the most
suitable candidate to support a distributed, networked audio implementation.
