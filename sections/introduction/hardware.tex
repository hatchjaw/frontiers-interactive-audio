\subsection{Hardware Platforms}\label{subsec:hardware-platforms}

The notion of taking a distributed approach to DSP is reliant on the
identification of a suitable supporting hardware platform.
For a distributed audio application, the ideal computing platform should be
small, inexpensive, plus easily and rapidly programmable;
of course, it should also provide audio and networking hardware, and, ideally,
well-documented APIs for programming and interacting with this hardware.

Recent years have seen the emergence of a number of small, low-cost, open-source
platforms for embedded development, perhaps best known amongst these being
Arduino\footnote{\url{https://arduino.cc/} \textemdash{} All URLs verified
12/01/2024}, which, with it's networking support, may be considered a range of
\textit{Internet of Things} (IoT) devices.
Though support for audio is limited via official Arduino models, a number of
audio-specific, Arduino-like systems have been produced, such as various
ESP32 and STM32 models, Daisy Seed (itself an STM32-based board), and the Teensy
family of microcontroller development boards.
Also worthy of consideration is the Bela platform; though not a true
microcontroller (being based on the BeagleBone single-board computer, and
running an operating system) it is a small-footprint audio-focused device
suitable for embedded applications.

A wealth of hardware add-ons (e.g.\ audio and ethernet boards, sensors,
microphones, etc.) and open source software libraries exist for Arduino- and
BeagleBone-based systems.
Bela itself is an audio ``cape'' for the BeagleBone Black; Teensy can be
augmented with a dedicated ``audio shield,'' and various digital-to-analog
converter breakout-boards exist for other Arduino-compatible devices.
Facilitating audio development, many such platforms are programmable with
Faust, a domain-specific programming language for audio synthesis and signal
processing\footnote{\url{https://faust.grame.fr/}}.
Faust provides \textit{architectures} for a number of embedded platforms via its
\texttt{faust2[...]} tools\footnote{
    \url{https://faustdoc.grame.fr/manual/tools/}
}, enabling users to write high level DSP code and compile it to meet the
requirements of the audio API on the target device~\citep{michon_real_2019,
    michon_embedded_2020}.

\begin{table}[t]
    \centering
    \begin{tabular}{ c c c r }
        Platform &
        Processor &
        Memory &
        Price \\

        \midrule

        Teensy 4.1\tablefootnote{\url{https://pjrc.com/store/teensy41.html}} &
        ARM Cortex-M7 \qty{600}{\MHz} &
        \qty{1}{\mega\byte} SDRAM &
        \texteuro{32} \\

        Daisy Seed\tablefootnote{\url{https://electro-smith.com/daisy/daisy}} &
        ARM Cortex-M7 \qty{480}{\MHz} &
        \qty{64}{\mega\byte} SDRAM &
        \texteuro{28} \\

        ESP32-LyraTD\tablefootnote{\url{https://espressif.com/en/products/devkits/esp-audio-devkits}} &
        Dual core Xtensa LX6 \qty{240}{\MHz} &
        \qty{8}{\mega\byte} PSRAM &
        \texteuro{19} \\

        STM32H747I\tablefootnote{\url{https://st.com/en/evaluation-tools/stm32h747i-disco.html}} &
        ARM Cortex-M7 \qty{480}{\MHz} + Cortex-M4 \qty{240}{\MHz} &
        \qty{1}{\mega\byte} RAM &
        \texteuro{94} \\

        Bela\tablefootnote{\url{https://shop.bela.io/products/bela-starter-kit}} &
        ARM Cortex-A8 \qty{1}{\GHz}\tablefootnote{\url{https://beagleboard.org/black}} &
        \qty{512}{\mega\byte} DDR3 &
        \texteuro{190}
    \end{tabular}
    \caption{Comparison of selected embedded audio development platforms.
    Prices as of January 2024.}
    \label{tab:embedded-comparison}
\end{table}

A comparison of selected devices can be found in
\tabref{tab:embedded-comparison}.
Bela is significantly more powerful than the microcontroller systems, but it is
commensurately costly.PJRC, the
Daisy Seed is well-appointed with memory (which is important for DSP algorithms
featuring long delay-lines, for example), but does not feature ethernet support.
Teensy4.1, and the selected ESP32 and STM32 devices support networking via
ethernet add-ons, but the ESP32's CPU is underpowered, and the STM32 is
unfavourably-priced.
Though lacking in memory, Teensy's processor, low price, networking support, and
programmability via the \texttt{faust2teensy} tool, make it an attractive
candidate platform for a distributed, networked audio implementation.
Audio development for the Teensy is further enhanced by the \textit{Teensy
Audio Library} and associated Audio System Design Tool\footnote{
    \url{https://pjrc.com/teensy/gui/index.html}
}, a web-based, graphical programming environment in which the user may describe
an audio program diagrammatically, and export the result to C++ to be compiled
for the Teensy.
Further, thanks to the presence of a vibrant developer community, utilities such
as \textit{TyTools}\footnote{\url{https://koromix.dev/tytools}} exist, and can
be used to program multiple Teensy devices in a single command;
for a system distributed amongst many such devices, this is particularly useful.

One respect in which Teensy is found wanting is audio fidelity.
By default, its audio shield produces CD quality output (16-bit,
\qty{44.1}{\kHz}), which is not sufficient for what might be deemed modern,
high-quality audio, such as offered by Daisy Seed (24-bit, \qty{96}{\kHz}).
While Teensy's sampling rate can be increased, sample resolution is fixed by
the Teensy Audio Library.
This shortcoming notwithstanding, and in light of its other, more advantageous
qualities, Teensy is the platform upon which development will proceed.
