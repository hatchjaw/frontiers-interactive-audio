\subsection{Distributed Computing}\label{subsec:distributed-computing}

In the broadest terms, a distributed system is \textit{``a collection of
independent entities that cooperate to solve a problem that cannot be
individually solved''}~\citep{kshemkalyani_distributed_2011}.
In turn, the term \textit{distributed computing} simply describes a system of
computation that is distributed in
space~\citep{lamport_distributed_1990}.\footnote{
    There is a certain linguistic symmetry here with respect to spatial audio,
    but that is as far as the parallel goes.
}
At a low enough level, this effectively describes \textit{any} computer
system, such systems being composed of individual entities \textemdash{}
processors, memory, input and output devices, etc. \textemdash{} all
acting in cooperation.

At a higher level \textemdash{} that of a computer network, for example
\textemdash{} why take a distributed approach to computation?
As Kshemkalyani and Singhal describe~\citep{kshemkalyani_distributed_2011},
a variety of rationales exist for taking such an approach;
adapting these to the notion of distributing an audio spatialisation algorithm
across a computer network, most relevant are:

\textbf{Scalability}
Particularly so if taking advantage of a network protocol that supports
multicast or broadcast transmission.
Under a unicast model, such as that employed by JackTrip, streams of audio data
are duplicated on a per-client basis;
under such a model, at some point all available network bandwidth will be
exhausted.
An ideal multicast networked audio system entails there being just one stream
of audio data, plus perhaps a stream of control data, for all clients to
consume.
Dependent on the application, the server in such a system may not need to be
aware of how many clients are connected; similarly, clients exist in isolation,
and fulfil their task with no dependency on their peers on the network.

\textbf{Modularity}
Closely related to scalability, modularity entails \textit{extensibility}.
This is something that centralised audio spatialisation systems either lack
entirely, or possess only at great cost in terms of hardware, and even then only
if the hardware supports extension via daisy-chaining, for example.
%The matter of expense anticipates:\textemdash{}

\textbf{Improved cost/performance ratio}
A modular, scalable system can be constructed to meet the proportion that
circumstances require, with the minimum amount of redundancy.
If it becomes desirable to scale the system up, this can be achieved by small
increments rather than by expensive leaps.\\

Kshemkalyani and Singhal also vaunt \textit{enhanced reliability} as motivation
for distributed computing.
This may indeed hold true for systems where the failure of a single node can be
compensated for by increased work on the part of the remaining nodes until such
time as the failed node can resume operation or be replaced;
it does not for the sort of system under consideration here.
Indeed, this points toward two significant drawbacks of distributed systems in
general: increased complexity and a proliferation of potential points of
failure.
Nodes in a distributed computational system must be served at the very least
with power and access (e.g.\ over a network) to the data that they require in
order to operate.
This entails the provision of cables, perhaps batteries, and physical
connections that may be subject to wear-and-tear or misuse.
Additional concerns surround the programmability of such a system, which is the
other side to the coin of modularity;
integrity, in terms of all nodes possessing up-to-date instructions for
operation, may be difficult to ensure.

\subsection{Distributed Audio Systems}\label{subsec:distributed-audio-systems}

The notion of taking a distributed approach to audio processing is by no means
unprecedented;
a selection of prior work in distributed DSP and audio spatialisation, plus
systems incorporating microcontrollers and single-board computers is detailed
below.

\subsubsection{State of the Art}

Applications of SoundWIRE to what its creators termed \textit{Internet
Acoustics}~\citep{chafe_physical_2002} obviously represent a case of distributed
audio processing.
These include implementations such as a network
reverberator~\citep{chafe_i_2018}, or
\textit{``transcontinental echo chamber''}~\citep{chafe_simplified_2000},
plus the aforementioned \textit{Network Harp}.
Experiments of this sort were intended initially as sonifications of QoS
\textemdash{} a characteristic of network systems that is difficult to represent
in graphical or textual form due to the ephemeral nature of phenomena such as
jitter and packet loss \textemdash{} but stand as fascinating applications in
their own right of digital audio in the age of computer networking.
Subsequent work on JackTrip has focused on optimising networked audio less
for sound processing or as a creative tool in itself, and more in service of
the social and communal aspects of music participation and appreciation in a
networked world;
these are topics that came to the fore in computer music research during the
COVID 19 pandemic~\citep{bosi_experiencing_2021,sacchetto_jacktrip-webrtc_2021}.

Lago~\citep{lago_distributed_2004} proposed a UDP-based system for real-time
distributed audio processing taking the form of a network of general purpose
computers.
A server sent packets of audio data to be processed by a collection of
clients, which would then return processed audio to the server to be combined
and used for output.
Since clients were not to be used directly for output, synchronisation was not
important, but Lago identifies the timing or hardware based interrupts for
audio and network processing as being of great importance to a distributed
real-time implementation.
Though an interesting exploration of approaching certain difficulties of
distributed computing, DSP in particular, and ambitious for its time (2004),
arguably the need for such a system has been obviated by advances in computer
processing power over the succeeding two decades.

A digital music production system of networked Beagleboard single-board
computers was demonstrated by Gabrielli et al.~\citep{gabrielli_networked_2012}
Another ambitious project, particularly since it relied on wireless
communication, the authors describe an interesting way of measuring
transmission round-trip times.\todo[inline]{See section about test signals?}

Exploring the possibilities of burgeoning network technology in the early 2010s,
Lopez-Lezcano set out to build a UDP-based `network sound card' to support
a networked WFS system~\citep{lopez-lezcano_jack_2012}.
The aim was to replace otherwise expensive high channel-count conventional
audio interfaces, which receive audio over the MADI (Multichannel Audio Digital
Interface) protocol, with a more cost-effective alternative.
Ultimately the devised system was not used for audio spatialisation, but
facilitated networked musical performance, and it stands as an example of the
results that can be achieved by using `raw' UDP data for audio transmission,
rather than an established protocol or system.

In addition to Gabrielli et al., implementations on IoT-like devices include
Chafe and Oshiro's port of JackTrip to the Raspberry Pi single-board computer
for further internet acoustics, plus distributed spatialisation systems such as
those described by Devonport and Foss~\citep{devonport_distribution_2019} and
Belloch et al.~\citep{belloch_performance_2021}
The latter two address aims closely aligned with the work described here, but
are based on costly computing platforms.
Devonport and Foss achieved high synchronicity via AVB; Belloch et al. used a
GPU-based hardware platform, which is perhaps unsuited to its task, and report
client synchronisation to the millisecond range \textemdash{} likely not
sufficient for timing-critical audio spatialisation effects.

TU Berlin.
Tons of money.
Dante.
800+ channels.

Also of interest is the OTTOsonics~\citep{mitterhuber_ottosonics_2022}
project;
its emphasis on a fully-costed, do-it-yourself alternative to conventional
spatial audio systems is pertinent to this work, though it diverges in its use
of AVB, and associated hardware for audio transmission.

\subsection{Challenges}\label{subsec:challenges}

Time, especially when dealing with the fine margins posed by real-time audio
processing, represents the principal source of difficulty in a distributed
audio setting.

\textit{Jitter} refers to fluctuations in the rate of transmission or
processing.
In a networked audio setting, jitter gives rise to a situation whereby the
arrival of audio data does not correspond with the moments at which it is
needed.
In a naive implementation, this may result in a recipient either halting
processing until it receives the expected data, or simply continuing without
any data.
In either case, the result is likely to be disruption of the integrity of the
audio signal in the form of audible discontinuities.

\textit{Clock drift} arises as an inevitable consequence of no source of time
in a system of computation being perfectly uniform, and no two sources of time
being identical.
The timing of a computer system is typically governed by a crystal oscillator,
the accuracy of which is affected by factors such as ambient temperature, and
potentially computational load on the system it
governs~\citep{marouani_internal_2008}.
Relative drift (sometimes, within the diagnostic parts of JackTrip for example,
referred to as \textit{skew}), is the difference in clock rates between two or
more systems.
Whereas jitter is a short-term phenomenon, clock drift typically takes effect
over a longer timescale.
As two distinct systems of time move in and out of phase with each other over
the longer term, drift may indeed give rise to jitter.

In studio and professional audio settings, devices may be synchronised via an
authoritative clock source such as word clock, or, in a networked setting,
via PTP or lower-resolution network time protocol.
In the absence of such an authoritative source, e.g.\ over a wide area network,
or if using hardware that does not support such measures, buffering strategies
are typically employed, coupled with delay-locked
loops~\citep{adriaensen_using_2005} and
resampling~\citep{adriaensen_controlling_2012}.
