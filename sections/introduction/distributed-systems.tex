\subsection{Distributed Audio Systems}\label{subsec:distributed-audio-systems}

In the broadest terms, a distributed system is \textit{``a collection of
independent entities that cooperate to solve a problem that cannot be
individually solved''}~\citep{kshemkalyani_distributed_2011}.
In turn, the term \textit{distributed computing} simply describes a system of
computation that is distributed in
space~\citep{lamport_distributed_1990}.

A variety of rationales exist for taking a distributed approach to
computation;
%adapting these to the notion of distributing an audio spatialisation algorithm
%across a computer network,
most relevant to the problem of distributing an audio spatialisation algorithm
across a computer network are:

\textbf{Scalability}
Particularly so if taking advantage of a network protocol that supports
multicast or broadcast transmission.
Under a unicast model, as employed by JackTrip, streams of audio data are
duplicated on a per-client basis;
under such a model, at some point all available network bandwidth will be
exhausted.
An ideal multicast networked audio system entails there being just one stream
of audio data, plus perhaps a stream of control data, for all clients to
consume.
Dependent on the application, the server in such a system may not need to be
aware of how many clients are connected; similarly, clients exist in isolation,
and fulfil their task with no dependency on their peers on the network.

\textbf{Modularity}
Closely related to scalability, modularity entails \textit{extensibility}.
This is something that centralised audio spatialisation systems either lack
entirely, or possess only at great cost in terms of hardware, and even then only
if the hardware supports extension via daisy-chaining, for example.

\textbf{Improved cost/performance ratio}
A modular, scalable system can be constructed to meet the proportion that
circumstances require, with the minimum amount of redundancy.
If it becomes desirable to scale the system up, this can be achieved by small
increments rather than by expensive leaps.

Where a distributed system may suffer when compared with its centralised
equivalent is in terms of reliability.
Nodes in a distributed computational system must be served with power and access
to the data they require in order to operate.
This entails the provision of power, cables and a proliferation of potential
points of failure.
The other side to the coin of modularity is a concern regarding the
programmability of such a system;
ensuring that all nodes possess up-to-date instructions for operation may not be
trivial.

Distributed audio processing is by no means a matter without precedent;
a selection of prior work in distributed DSP and audio spatialisation, plus
systems incorporating microcontrollers and single-board computers is detailed
below.

\subsubsection{State of the Art}

Applications of SoundWIRE to what its creators termed \textit{Internet
Acoustics}~\citep{chafe_physical_2002} obviously represent a case of distributed
audio processing.
These include a network reverberator~\citep{chafe_i_2018}, or
\textit{``transcontinental echo chamber''}~\citep{chafe_simplified_2000},
plus the aforementioned \textit{Network Harp}.
Experiments of this sort were intended initially as sonifications of QoS
\textemdash{} a characteristic of network systems that is difficult to represent
in graphical or textual form due to the ephemeral nature of the phenomena of
jitter and packet loss \textemdash{} but stand as fascinating applications in
their own right of digital audio in the age of computer networking.
Subsequent work on JackTrip has focused on optimising networked audio less
for sound processing or as a creative tool in itself, and more in service of
the social and communal aspects of music participation and appreciation in a
networked world;
these are topics that came to the fore in computer music research during the
COVID 19 pandemic~\citep{bosi_experiencing_2021,sacchetto_jacktrip-webrtc_2021}.

Lago~\citep{lago_distributed_2004} proposed a UDP-based system for real-time
distributed audio processing taking the form of a network of general purpose
computers.
A server sent packets of audio data to be processed by a collection of
clients, which would then return processed audio to the server to be combined
and used for output.
Since clients were not to be used directly for output, synchronisation was not
important, but Lago identifies the timing or hardware based interrupts for
audio and network processing as being of great importance to a distributed
real-time implementation.
Though an interesting exploration of approaching certain difficulties of
distributed computing, DSP in particular, and ambitious for its time (2004),
arguably the need for such a system has been obviated by advances in computer
processing power over the succeeding two decades.

A digital music production system of networked Beagleboard single-board
computers was demonstrated by Gabrielli et al.~\citep{gabrielli_networked_2012}
Another ambitious project, particularly since it utilised wireless
communication, an example configuration consisted of three nodes, one sending
control data to another, which produced audio and transmitted it to the third,
which recorded it in a Digital Audio Workstation (DAW).
The authors were more concerned with latency than synchronicity for their
purposes, and measured latency via transmission round-trip times using a
sawtooth wave as a timer (see \secref{subsec:technical-evaluation} for an
application of this technique).

Exploring the possibilities of burgeoning network technology in the early 2010s,
Lopez-Lezcano set out to build a UDP-based `network sound card' to support
a networked WFS system~\citep{lopez-lezcano_jack_2012}.
The aim was to replace otherwise expensive high channel-count conventional
audio interfaces, which receive audio over the MADI (Multichannel Audio Digital
Interface) protocol, with a more cost-effective alternative.
Ultimately the devised system was not used for audio spatialisation, but
facilitated networked musical performance, and it stands as an example of the
results that can be achieved by using `raw' UDP data for audio transmission,
rather than an established protocol or system.

In addition to Gabrielli et al., implementations on IoT-like devices include
Chafe and Oshiro's port of JackTrip to the Raspberry Pi single-board computer
for further internet acoustics, plus distributed spatialisation systems such as
those described by Devonport and Foss~\citep{devonport_distribution_2019} and
Belloch et al.~\citep{belloch_performance_2021}
The latter two address aims closely aligned with the work described here, but
are based on costly computing platforms.
Devonport and Foss achieved high synchronicity via AVB; Belloch et al.\ used a
GPU-based hardware platform, which is perhaps unsuited to its task, and report
client synchronisation to the millisecond range \textemdash{} likely not
sufficient for timing-critical audio spatialisation effects.

Also of interest is the OTTOsonics~\citep{mitterhuber_ottosonics_2022}
project;
its emphasis on a fully-costed, flexible, do-it-yourself alternative to
conventional spatial audio systems is pertinent to this work, though it diverges
in its use of AVB, and associated hardware for audio transmission.

A full 24-channel OTTOsonics system, including speakers and audio interface, is
costed at around \texteuro{2600}.
A more conventional, 64-channel WFS system, such as is installed in the
Multisensory Experience Lab at Aalborg University (AAU),
Copenhagen~\citep{grani_gestural_2016}, features two 32-channel MADI to analogue
converters costing roughly \texteuro{5000} apiece.
The world's largest WFS system, at TU Berlin, features over 800 output channels
driven by a distributed cluster of fifteen
computers~\citep{baalman_renewed_2007}\footnote{
    See also
    \url{https://tu.berlin/en/ak/research/projects/wellenfeldsynthese-fuer-einen-grossen-hoersaal}
    and WFS speaker module produced by Four Audio for installation at TU
    \url{https://four-audio.com/en/products/wfs/}.
}; the expense associated with this system is difficult to assess, as it is tied
to the concert hall space in which it resides.
Indeed, one thing besides expense that unites the systems at AAU and TU Berlin,
plus similar installations at IRCAM (Paris)\footnote{
    \url{https://ircam.fr/article/connaissez-vous-lespace-de-projection}
} and the Rensselaer Polytechnic Institute (NY, USA)\footnote{
    \url{https://empac.rpi.edu/about/building/venues}
} is their \textit{in-situ} nature;
these are site-specific systems with, at best, limited flexibility.

\subsubsection{Challenges}\label{subsubsec:challenges}

Time, especially when dealing with the fine margins posed by real-time audio
processing, represents the principal source of difficulty in a distributed
audio setting.

\textit{Jitter} refers to fluctuations in the rate of transmission or
processing.
In a networked audio setting, jitter gives rise to a situation whereby the
arrival of audio data does not correspond with the moments at which it is
needed.
In a naive implementation, this may result in a recipient either halting
processing until it receives the expected data, or simply continuing without
any data.
In either case, the result is likely to be disruption of the integrity of the
audio signal in the form of audible discontinuities.

\textit{Clock drift} arises as an inevitable consequence of no source of time
in a system of computation being perfectly uniform, and no two sources of time
being identical.
The timing of a computer system is typically governed by a crystal oscillator,
the accuracy of which is affected by factors such as ambient temperature, and
potentially computational load on the system it
governs~\citep{marouani_internal_2008}.
Relative drift (sometimes, within the diagnostic parts of JackTrip for example,
referred to as \textit{skew}), is the difference in clock rates between two or
more systems.
Whereas jitter is a short-term phenomenon, clock drift typically takes effect
over a longer timescale.
As two distinct systems of time move in and out of phase with each other over
the longer term, drift may indeed give rise to jitter.

In studio and professional audio settings, devices may be synchronised via an
authoritative clock source such as word clock, or, in a networked setting,
via PTP or lower-resolution network time protocol.
In the absence of such an authoritative source, e.g.\ over a wide area network,
or if using hardware that does not support such measures, buffering strategies
are typically employed, coupled with delay-locked
loops and resampling~\citep{adriaensen_using_2005, adriaensen_controlling_2012}.
