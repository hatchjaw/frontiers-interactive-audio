\section{Discussion}\label{sec:discussion}

The temporal clustering and polarisation seen in figures~\ref{fig:rtt-drift-16}
and~\ref{fig:rtt-drift-32} is indicative of two points for concern with regard
to technical implementation:
the read-write difference threshold strategy may be insufficiently forgiving,
forcing the read position into deleteriously fluctuating increment changes in
response to periods of jitter.
And without a master clock to indicate to each client the beginning of each
output audio block, even with clock rates perfectly aligned, there is nothing
to guarantee agreement of the timing of audio interrupts at the client side.

To avoid sudden, large or `unrealistic' clock adjustments, clients assess
the drift ratio as assessed via the ratio of network packet
transmission to reception and, if it lies beyond an arbitrary threshold,
simply resets the clock to the default \qty{44.1}{\kHz}.
Resets of this sort may account for the large steps seen in the drift plots
in Figures~\ref{fig:rtt-drift-16} and~\ref{fig:rtt-drift-32}.
It is clear that this strategy could be significantly improved upon.

The phasing effect noted by one participant is a disappointing consequence
of the approach taken to combating jitter and keeping the clients close,
temporally, together, and as close to the server as possible.
It is clear that the current approach is, at best, too aggressive to be viable
for high-quality audio output.
Furthermore, an unpitched sound source such as a snare drum, though audibly
susceptible to the time-varying comb-filter effect described, masks other
artefacts caused by phenomena such as rapid fluctuations in the clients' buffer
read position increment, and sudden, comparatively large audio clock
adjustments.

\subsection{Conclusion}\label{subsec:conclusion}

This article has described an exploration of certain fundamental aspects of
digital audio, transmission of audio over computer networks, distributed
computing and audio spatialisation.
A networked audio system was developed, strategies devised for
addressing challenges presented by distributed computing,
and a distributed spatial audio system was developed and deployed.
Evaluation of the system exposed the extent of the technical challenges
that confront it in its current form, and shed light on opportunities for
further development.
In an informal setting, perceptual testing revealed that it may, with heavy
provisos, offer performance sufficient to support timing-critical sound field
synthesis techniques.

It is pertinent at this point to consider the research questions stated in
section~\ref{s:research-questions}, and what exploring those questions has
revealed.

\paragraph{Research Question 1}
It has been shown that a system of discrete computational entities can
communicate effectively over an ethernet network, exchanging audio and control
data in a scalable fashion via a UDP multicast group.
Though facing significant technical challenges, an approach to the foundational
requirements of such a system has been demonstrated, with encouraging results
with regard to spatial audio applications.

\paragraph{Research Question 2}
Timing discrepancies between entities in the networked audio system give rise
to audible artefacts such as time-varying comb-filtering.
Loss of synchronicity is difficult to measure and compensate for in real-time
and the extent of the resulting audible disturbances is unpredictable, but,
depending on the nature of the sound sources being dealt with,
certainly perceptible.
The developed strategies for mitigation of asynchronicity are
\textit{best-effort} in nature, and call for further refinement, or replacement
with more sophisticated techniques.

\subsection{Future Work}\label{subsec:future-work}

Given its potential to disrupt the present situation with regard to spatial
audio installations, or complement it with a modular approach that could serve
more flexible and perhaps creative ends, it is hoped that scope will exist to
develop this work further.
A number of technical challenges remain, and questions with regard to the
fundamental, low-level characteristics of the various components of the devised
system stand unanswered.
From the level of audio hardware and driver software, to audio host and audio
buffer behaviour, to network QoS and the performance of network switches, to
the behaviour of the Teensy platform, and the processor and audio codec that it
is built around, plus its software libraries \textemdash{} much that is
typically taken for granted in the development of embedded systems, and audio
and networking systems, calls for deeper investigation.

\paragraph{Clock-sharing}
Each Teensy generates a clock signal to deliver to its audio shield.
In a locally distributed setting, this is quite redundant.
It may be possible to generate a single, authoritative clock on one Teensy
and deliver that to all others in the system.
If this can be achieved, it would remedy the issue of clock drift, leaving
only jitter to be addressed.

\paragraph{Further Audio Spatialisation Techniques}
A basic, linear, distributed wave field synthesis algorithm has been
demonstrated, implementing primary virtual sources.
WFS is capable of producing other types of sources, supporting nonlinear
speaker arrays, three-dimensional arrays, and more faithful models of
energy loss due to absorption.
There are many further avenues to pursue, including optimising any DSP
algorithms to best exploit the capabilities of what is, in the shape of the
Teensy, a very powerful platform.
Further, and only given cursory treatment here, higher order ambisonics remains
as a worthy target for implementation in future work.
